\documentclass[10pt,a4paper]{article}

\usepackage[utf8x]{inputenc}
\usepackage{ucs}
\usepackage{amsmath}
\usepackage{amsfonts}
\usepackage{amssymb}
\usepackage{fullpage}
\usepackage[slovene]{babel}
\usepackage{graphicx}
\usepackage{multirow}
\usepackage{tabularx}
\usepackage{ifpdf}
\usepackage{listings}
\usepackage{xcolor}
\usepackage{hyperref}

\definecolor{Butter}{rgb}{0.93,0.86,0.25}
\definecolor{Orange}{rgb}{0.96,0.47,0.00}
\definecolor{Chocolate}{rgb}{0.75,0.49,0.07}
\definecolor{Chameleon}{rgb}{0.45,0.82,0.09}
\definecolor{SkyBlue}{rgb}{0.20,0.39,0.64}
\definecolor{LightBlue}{rgb}{0.80,0.80,0.99}
\definecolor{DarkSkyBlue}{rgb}{0.10,0.19,0.34}
\definecolor{Plum}{rgb}{0.46,0.31,0.48}
\definecolor{Aluminium4}{rgb}{0.46,0.46,0.48}
\definecolor{ScarletRed}{rgb}{0.80,0.00,0.00}
\definecolor{DarkScarletRed}{rgb}{0.40,0.00,0.00}

\lstset{
	numbers=left,                   % where to put the line-numbers
	numberstyle={\small},      % the size of the fonts that are used for the line-numbers
	keywordstyle=[1]{\color{DarkSkyBlue}},
	keywordstyle=[2]{\color{DarkScarletRed}},
	keywordstyle=[3]{\bfseries},
	keywordstyle=[4]{\color{DarkPlum}},
	keywordstyle=[5]{\color{SkyBlue}},
	commentstyle={\color{Aluminium4}},
	stringstyle={\color{Chocolate}},
	basicstyle={\ttfamily\small},
	xleftmargin=17pt,
	breaklines=true,
	inputencoding=utf8x, 
	extendedchars=\true,
	frame=single
}



\pagestyle{empty}

\begin{document}

\begin{titlepage}
\begin{center}

% Upper part of the page
\includegraphics{Screenshot.png}\\[4.0cm]    
\textsc{\Large Dokumentacija}\\[4.5cm]

% Title
\hrule \ \\[0.2cm]
{ \huge \bfseries Navidezna fakulteta v WebGLu}\\[0.3cm]
\hrule \ \\[4.5cm]

% Author and supervisor
\begin{minipage}{0.4\textwidth}
\begin{flushleft} \large
\emph{Avtorji:}\\
Miha \textsc{Zidar}\\
Anže \textsc{Pečar}\\
Aleksandra \textsc{Bersan}\\
\end{flushleft}
\end{minipage}
\begin{minipage}{0.4\textwidth}
\begin{flushright} \large
\end{flushright}
\end{minipage}
\vfill

% Bottom of the page
{\large \today}
\end{center}
\end{titlepage}
\pagebreak
\section{Opis rešitve}
\subsection{Kaj rešitev sploh je}
Vsi vemo, da ima naša fakulteta zelo nelogično razporejene prostore in pogosto 
se zgodi, da obiskovalec ne najde prave predavalnice. Da bi obiskovalcem in 
brucom olajšali življenje, smo se odločili narediti 3-D model fakultete.\\\\
Ker želimo, da bi bila naša aplikacija čim lažje dostopna, smo jo postavili na 
internet s pomočjo odprtokodne knjižnice WebGL. Potrudili smo se, da aplikacija 
teče tekoče na sodobnih brskalnikih z WebGL podporo.
\subsection{Uporabljene metode}
\subsection*{Jeziki}
\begin{itemize}
	\item JavaScript - Delo z gl knjižnico
	\item GLSL - za Fragment in Vertex shaderja
	\item Python - Backend, pretvarjanje .obj datotek v json
\end{itemize}
\subsection*{Ogrodja}
\begin{itemize}
	\item webql-utils - Za funkcijo requestAnimFrame
	\item glMatrix - Uporabne funkcije nad matrikami in vektorji
	\item jQuery - Izboljšanje uporabniške iskušnje
	\item Django - Backend za shranjevanje podatkov o mestih zanimanja v podatkovno bazo 
\end{itemize}
\subsection*{Orodja}
\begin{itemize}
	\item Blender - Modeliranje faksa po gradbenih načrtih
	\item Gimp - Izdelava tekstur
\end{itemize}
%\subsection*{Metode}
\subsection*{Algoritmi}
\begin{itemize}
	\item Mipmap teksture
	\item Phongov odbojni model
	\item Collision detection
\end{itemize}
\begin{center}
\includegraphics{./WebGL.png}
\end{center}
\pagebreak
\section{Zabavni deli}
\subsection{Collision detection}
\subsection{Pretvarjanje iz .obj v json in parsanje}
\subsection{Phongov odbojni model}
\pagebreak
\section{Problemi}
\subsection{Blacklistane graficne}
\subsection{Lepljenje tekstur}
\subsection{Nesreca enega clana ekipe}
\pagebreak
\section{Posnetki zaslonov}
\end{document}  %End of document.
