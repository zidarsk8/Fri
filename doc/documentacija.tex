\documentclass[10pt,a4paper]{article}

\usepackage[utf8x]{inputenc}
\usepackage{ucs}
\usepackage{amsmath}
\usepackage{amsfonts}
\usepackage{amssymb}
\usepackage{fullpage}
\usepackage[slovene]{babel}
\usepackage{graphicx}
\usepackage{multirow}
\usepackage{tabularx}
\usepackage{ifpdf}
\usepackage{listings}
\usepackage{xcolor}
\usepackage{hyperref}

\definecolor{Butter}{rgb}{0.93,0.86,0.25}
\definecolor{Orange}{rgb}{0.96,0.47,0.00}
\definecolor{Chocolate}{rgb}{0.75,0.49,0.07}
\definecolor{Chameleon}{rgb}{0.45,0.82,0.09}
\definecolor{SkyBlue}{rgb}{0.20,0.39,0.64}
\definecolor{LightBlue}{rgb}{0.80,0.80,0.99}
\definecolor{DarkSkyBlue}{rgb}{0.10,0.19,0.34}
\definecolor{Plum}{rgb}{0.46,0.31,0.48}
\definecolor{Aluminium4}{rgb}{0.46,0.46,0.48}
\definecolor{ScarletRed}{rgb}{0.80,0.00,0.00}
\definecolor{DarkScarletRed}{rgb}{0.40,0.00,0.00}

\lstset{
	numbers=left,                   % where to put the line-numbers
	numberstyle={\small},      % the size of the fonts that are used for the line-numbers
	keywordstyle=[1]{\color{DarkSkyBlue}},
	keywordstyle=[2]{\color{DarkScarletRed}},
	keywordstyle=[3]{\bfseries},
	keywordstyle=[4]{\color{DarkPlum}},
	keywordstyle=[5]{\color{SkyBlue}},
	commentstyle={\color{Aluminium4}},
	stringstyle={\color{Chocolate}},
	basicstyle={\ttfamily\small},
	xleftmargin=17pt,
	breaklines=true,
	inputencoding=utf8x, 
	extendedchars=\true,
	frame=single
}



\pagestyle{empty}

\begin{document}

\begin{titlepage}
\begin{center}

% Upper part of the page
\includegraphics{Screenshot.png}\\[4.0cm]    
\textsc{\Large Dokumentacija}\\[4.5cm]

% Title
\hrule \ \\[0.2cm]
{ \huge \bfseries Navidezna fakulteta v WebGLu}\\[0.3cm]
\hrule \ \\[4.5cm]

% Author and supervisor
\begin{minipage}{0.4\textwidth}
\begin{flushleft} \large
\emph{Avtorji:}\\
Miha \textsc{Zidar}\\
Anže \textsc{Pečar}\\
Aleksandra \textsc{Bersan}\\
\end{flushleft}
\end{minipage}
\begin{minipage}{0.4\textwidth}
\begin{flushright} \large
\end{flushright}
\end{minipage}
\vfill

% Bottom of the page
{\large \today}
\end{center}
\end{titlepage}
\pagebreak
\section{Opis rešitve}
\subsection{Kaj rešitev sploh je}
Vsi vemo, da ima naša fakulteta zelo nelogično razporejene prostore in pogosto 
se zgodi, da obiskovalec ne najde prave predavalnice. Da bi obiskovalcem in 
brucom olajšali življenje, smo se odločili narediti 3-D model fakultete.\\\\
Ker želimo, da bi bila naša aplikacija čim lažje dostopna, smo jo postavili na 
internet s pomočjo odprtokodne knjižnice WebGL. Potrudili smo se, da aplikacija 
teče tekoče na sodobnih brskalnikih z WebGL podporo.
\subsection{Uporabljene metode}
\begin{center}
\includegraphics{./WebGL.png}
\end{center}
\subsection*{Jeziki}
\begin{itemize}
	\item \verb|JavaScript| - Delo z gl knjižnico
	\item \verb|GLSL| - za Fragment in Vertex shaderja
	\item \verb|Python| - Backend, pretvarjanje .obj datotek v json
\end{itemize}
\subsection*{Ogrodja}
\begin{itemize}
	\item \verb|webql-utils| - Za funkcijo requestAnimFrame
	\item \verb|glMatrix| - Uporabne funkcije nad matrikami in vektorji
	\item \verb|jQuery| - Izboljšanje uporabniške iskušnje
	\item \verb|Django| - Backend za shranjevanje podatkov o mestih zanimanja v podatkovno bazo 
\end{itemize}
\subsection*{Orodja}
\begin{itemize}
	\item Blender - Modeliranje faksa po gradbenih načrtih
	\item Gimp - Izdelava tekstur
\end{itemize}
%\subsection*{Metode}
\subsection*{Algoritmi}
\begin{itemize}
	\item Mipmap teksture
	\item Phongov odbojni model
	\item Collision detection
\end{itemize}
\pagebreak
\section{Zabavni deli}
\subsection{Pretvarjanje iz .obj v json in parsanje}
Za izrisovanja modelov iz Blenderja je potrebno prebrati datoteko v .obj formatu in iz zapisa
izluščiti $vertexe$, $normale$, $face$ in različne materiale. Za ta namen smo 
napisali preprosto python skripto, ki .obj datoteko prebere in prebrane podatke 
zapiše v json formatu. Na ta način smo pohitrili začetek izrisovanja, saj spletnemu
brskalniku ni potrebno parsati .obj datoteke ob vsaki naložitvi strani.
\subsection{Phongov odbojni model}
\subsection{Collision detection}
\pagebreak
\section{Problemi}
\subsection{Blacklisted GPUs}
Še preden smo začeli z razvojem naše seminarske naloge, smo naleteli na probleme.
Tako Googlov Chrome, kot Mozillin Firefox sta imela na črni listi grafične kartice,
ki so bile v naših računalnikih, kar je pomenilo, da je bil WebGL onemogočen. 
WebGLa ni bilo možno prisiliti k delovanju iz nastavitev znotraj brskalnika pa 
čeprav smo šarili po nastavitvah za developerje (Chromov: \verb|about:flags| in FFjev: \verb|about:config|).
Po nekaj dneh Googlanja smo le naleteli na sistemsko spremenljivko \verb|MOZ_GLX_IGNORE_BLACKLIST=1|,
s katero smo omogočili WebGL znotraj FireFoxa in Chromovo \verb|--ignore-gpu-blacklist| zagonsko stikalo.\\\\
Z najnovejšo različico Google Chromea nastavljanje zagonskega stikala ni več potrebno, 
FireFox4 pa še vedno potrebuje nastavljeno sistemsko spremenljivko.
\subsection{Lepljenje tekstur}
\subsection{Zaplet v ekipi}
Zaradi hujše prometne nesreče članica ekipe Aleksandra ni mogla opraviti svojega dela 
seminarske naloge v celoti, zato sva Anže in Miha sama nadaljevala projekt. 
\pagebreak
\section{Posnetki zaslonov}
\end{document}  %End of document.
