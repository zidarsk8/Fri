\documentclass{article}
\usepackage{times}
\usepackage[utf8]{inputenc}
\usepackage{graphicx}
\usepackage{listings}
\usepackage[slovene]{babel}
\usepackage{hyperref}
\begin{document}
\lstset{frame=single,numbers=left,language=C,breaklines=true,basicstyle={\ttfamily \scriptsize}}
% Article top matter
\title{Seminarska naloga Navidezna fakulteta}
\author{Anže Pečar\\
        Aleksandra Bersan\\
        Miha Zidar\\\\
	    Fakulteta za računalništvo in informatiko,\\
		Tržaška cesta 25,\\
		1000 ljubljana\\}		
\date{\today}
\maketitle
\pagebreak
\section{Predstavitev}
Vsi vemo, da ima naša fakulteta zelo nelogično razporejene prostore in pogosto se zgodi, da obiskovalec ne najde prave predavalnice. Da bi obiskovalcem in brucom olajšali življenje,
smo se odločili narediti 3-D model fakultete.
\section{Tehnologija}
Ker želimo, da bi bila naša aplikacija čim lažje dostopna, jo bomo postavili na internet s pomočjo odprtokodne knjižnice WebGL. Potrudili se bomo, da bo aplikacija tekla tekoče na vseh sodobnih brskalnikih z WebGL podporo. Ker pa imajo različni brskalniki različni nivo podpore WebGL standarda, se bomo vsaj na začetku osredotočili na Firefox4. \\
\begin{center}
\includegraphics{./WebGL.png}
\end{center}
\section{Cilji}
Narediti želimo podroben model fakultete po katerem se bodo obiskovalci lahko sprehajali. Naš cilj je narediti preprost vmesnik, ki bo obiskovalcu pomagal najti hitro pot do željene predavalnice. Poleg informacij o predavalnicah bo naša aplikacija vsebovala tudi evakuacijske poti ter zasilne izhode. Ogledati si bo mogoče tudi moč brezžičnega omrežja EDUROAM v posameznih delih fakultete.


\end{document}  %End of document.
