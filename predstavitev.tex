\documentclass{article}
\usepackage{times}
\usepackage[utf8]{inputenc}
\usepackage{graphicx}
\usepackage{listings}
\usepackage[slovene]{babel}
\usepackage{hyperref}
\begin{document}
\lstset{frame=single,numbers=left,language=C,breaklines=true,basicstyle={\ttfamily \scriptsize}}
% Article top matter
\title{Seminarska naloga Navidezna fakulteta}
\author{Anže Pečar\\
        Aleksandra Bersan\\
        Miha Zidar\\\\
	    Fakulteta za računalništvo in informatiko,\\
		Tržaška cesta 25,\\
		1000 ljubljana\\}		
\date{\today}
\maketitle
\pagebreak
\section{Predstavitev}
Vsi vemo, da ima naša fakulteta zelo nelogično razporejene prostore in pogosto se zgodi, da obiskovalec ne najde prave predavalnice. Da bi obiskovalcem in brucom olajšali življenje,
smo se odločili narediti 3-D model fakultete.
\section{Tehnologija}
Ker želimo, da bi bila naša aplikacija čim lažje dostopna, bomo uporabili odprtokodno knjižnico WebGL. Tekom razvoja bomo naredili model, ki bo deloval na Firefoxu4.
\section{Cilji}
Narediti želimo podroben model fakultete po katerem se bodo obiskovalci lahko sprehajali in raziskovali. Prav tako se bo lahko izračunala najkrajša pot do 
predavalnice, ki jo iščejo. Model pa bo tudi vseboval podatke o moči signala brezžičnega omrežja EDUROAM v posameznih delih fakultete. 
\end{document}  %End of document.
