\documentclass[10pt,a4paper]{article}

\usepackage[utf8x]{inputenc}
\usepackage{ucs}
\usepackage{amsmath}
\usepackage{amsfonts}
\usepackage{amssymb}
\usepackage{fullpage}
\usepackage[slovene]{babel}
\usepackage{graphicx}
\usepackage{multirow}
\usepackage{tabularx}
\usepackage{ifpdf}
\usepackage{listings}
\usepackage{xcolor}
\usepackage{hyperref}

\definecolor{Butter}{rgb}{0.93,0.86,0.25}
\definecolor{Orange}{rgb}{0.96,0.47,0.00}
\definecolor{Chocolate}{rgb}{0.75,0.49,0.07}
\definecolor{Chameleon}{rgb}{0.45,0.82,0.09}
\definecolor{SkyBlue}{rgb}{0.20,0.39,0.64}
\definecolor{LightBlue}{rgb}{0.80,0.80,0.99}
\definecolor{DarkSkyBlue}{rgb}{0.10,0.19,0.34}
\definecolor{Plum}{rgb}{0.46,0.31,0.48}
\definecolor{Aluminium4}{rgb}{0.46,0.46,0.48}
\definecolor{ScarletRed}{rgb}{0.80,0.00,0.00}
\definecolor{DarkScarletRed}{rgb}{0.40,0.00,0.00}

\lstset{
	numbers=left,                   % where to put the line-numbers
	numberstyle={\small},      % the size of the fonts that are used for the line-numbers
	keywordstyle=[1]{\color{DarkSkyBlue}},
	keywordstyle=[2]{\color{DarkScarletRed}},
	keywordstyle=[3]{\bfseries},
	keywordstyle=[4]{\color{DarkPlum}},
	keywordstyle=[5]{\color{SkyBlue}},
	commentstyle={\color{Aluminium4}},
	stringstyle={\color{Chocolate}},
	basicstyle={\ttfamily\small},
	xleftmargin=17pt,
	breaklines=true,
	inputencoding=utf8x, 
	extendedchars=\true,
	frame=single
}



\pagestyle{empty}

\author{Miha Zidar (63060317)}
\title{OS1: Peta domača naloga}


\begin{document}

\begin{titlepage}
\begin{center}

% Upper part of the page
\includegraphics{Screenshot.png}\\[4.0cm]    
\textsc{\Large Seminarska naloga}\\[4.5cm]

% Title
\hrule \ \\[0.2cm]
{ \huge \bfseries Navidezna fakulteta v WebGLu}\\[0.3cm]
\hrule \ \\[4.5cm]

% Author and supervisor
\begin{minipage}{0.4\textwidth}
\begin{flushleft} \large
\emph{Avtorji:}\\
Miha \textsc{Zidar}\\
Anže \textsc{Pečar}\\
Aleksandra \textsc{Bersan}\\
\end{flushleft}
\end{minipage}
\begin{minipage}{0.4\textwidth}
\begin{flushright} \large
%\emph{Mentor:} \\
%dr. Rajko  \textsc{Mahkovic}\\
%Aleksander   \textsc{Lukič}
\end{flushright}
\end{minipage}
\vfill

% Bottom of the page
{\large \today}

\end{center}
\end{titlepage}
\pagebreak
\section{Predstavitev}
Vsi vemo, da ima naša fakulteta zelo nelogično razporejene prostore in pogosto se zgodi, da obiskovalec ne najde prave predavalnice. Da bi obiskovalcem in brucom olajšali življenje,
smo se odločili narediti 3-D model fakultete.
\section{Tehnologija}
Ker želimo, da bi bila naša aplikacija čim lažje dostopna, jo bomo postavili na internet s pomočjo odprtokodne knjižnice WebGL. Potrudili se bomo, da bo aplikacija tekla tekoče na vseh sodobnih brskalnikih z WebGL podporo. Ker pa imajo različni brskalniki različni nivo podpore WebGL standarda, se bomo vsaj na začetku osredotočili na Firefox4. \\
\begin{center}
\includegraphics{./WebGL.png}
\end{center}
\section{Cilji}
Narediti želimo podroben model fakultete po katerem se bodo obiskovalci lahko sprehajali. Naš cilj je narediti preprost vmesnik, ki bo obiskovalcu pomagal najti hitro pot do željene predavalnice. Poleg informacij o predavalnicah bo naša aplikacija vsebovala tudi evakuacijske poti ter zasilne izhode. Ogledati si bo mogoče tudi moč brezžičnega omrežja EDUROAM v posameznih delih fakultete.


\end{document}  %End of document.
